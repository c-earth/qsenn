\documentclass[preprint]{revtex4-2}
\usepackage{amssymb}
\usepackage{amsmath}
\DeclareMathOperator{\Tr}{Tr}
\usepackage{physics}


\begin{document}
\title{su2nn Irreducible Representation}
\author{Abhijatmedhi Chotrattanapituk}
\affiliation{Quantum Measurement Group, MIT, Cambridge, MA, USA \\
            Department of Electrical Engineering and Computer Science, MIT, Cambridge, MA, USA}

\date{\today}
\maketitle

\section{Introduction}
TBA

\section{$SU(2)$'s Definition}
$SU(2)$ is the special unitary group of dimension 2, i.e. the group of 2 by 2 unitary matrices 
with unit determinant. In other words,
\begin{equation}
    SU(2) = \{u \in GL(2, \mathbb{C})|u^\dagger=u^{-1}, \det(u) = 1\},
\end{equation}    
where $GL(2,\mathbb{C})$ is the group of 2 by 2 invertible matrices with complex field, and $u^\dagger$
is the complex conjugate transposition of $u$. Since for any $u, v \in SU(2)$
\begin{equation}
    \det(uv) = \det(u)\det(v) = 1,
\end{equation}
and
\begin{equation}
    (uv)^\dagger = (uv)^{\top\ast} = (v^\top u^\top)^\ast = v^\dagger u^\dagger = v^{-1}u^{-1} = (uv)^{-1},
\end{equation}
$SU(2)$ is closed subgroup of $GL(2, \mathbb{C})$ which makes it a Lie group. This means that
there is a corresponding Lie algebra, $\mathfrak{su(2)}$, such that 
\begin{equation}
    SU(2) = \{e^g|g \in \mathfrak{su(2)}\}.
\end{equation}

From definition of matrix exponential,
\begin{equation}
    \begin{aligned}
        (e^g)^\dagger &= e^{g^\dagger}, \\
        (e^g)^{-1} &= e^{-g}, \\
        \det(e^g) &= e^{\Tr g},
    \end{aligned}
\end{equation}
one can define the Lie algebra of $SU(2)$ as
\begin{equation}
    \mathfrak{su(2)} = \{g \in M(2, \mathbb{C})|g^\dagger=-g, \Tr(g) = 0\},
\end{equation}
where $M(2,\mathbb{C})$ is the group of 2 by 2 matrices with complex field, and $\Tr(g)$ is the
trace of matrix $g$. Hence, each member $g$ of $\mathfrak{su(2)}$ can be written in the From
\begin{equation}
    g = \begin{bmatrix}
            iv_z & v_y+iv_x \\
            -v_y+iv_x & -iv_z
        \end{bmatrix}
      = iv_x\sigma_x + iv_y\sigma_y + iv_z\sigma_z = i\vec{v}\cdot\vec{\sigma}
\end{equation}
where $\vec{v} \in \mathbb{R}^3$, and $\sigma_i$'s are Pauli matrices.

\section{$\mathfrak{su(2)}$'s Irreducible Representation}
From $\mathfrak{su(2)}$, one can consider Pauli matrices as the group generators, but it is more
common to use $J^{(1/2)}_i = \sigma_i/2$ as the generators. This gives the Lie bracket relation as
\begin{equation} \label{eq:su2_bracket}
    \left[J^{(1/2)}_i, J^{(1/2)}_j\right] = i\epsilon_{ijk}J^{(1/2)}_k.
\end{equation}
The next step is to consider an arbritary representation of $SU(2)$ such that its generators, $J_i$'s,
obey (\ref{eq:su2_bracket}). With out loss of generality, consider the representation vector space with eigenvectors of
$J_z$, $\{\ket{\lambda_{J_z}}\}$, as the basis,
\begin{equation}
    J_z\ket{\lambda_{J_z}}=\lambda_{J_z}\ket{\lambda_{J_z}}.
\end{equation}
Furthermore, we will also consider only the representation that is irreducible. With the standard method, first define 
two additional operators
\begin{equation}
    \begin{aligned}
        J_+ &= J_x + iJ_y, \\
        j_- &= J_x - iJ_y.
    \end{aligned}
\end{equation}
From Lie bracket, and the choice of basis used,
\begin{equation}
    \begin{aligned}
        J_zJ_+\ket{\lambda_{J_z}} &= J_+(J_Z+1)\ket{\lambda_{J_z}} = (\lambda_{J_z}+1)J_+\ket{\lambda_{J_z}}, \\
        J_zJ_-\ket{\lambda_{J_z}} &= J_-(J_Z-1)\ket{\lambda_{J_z}} = (\lambda_{J_z}-1)J_-\ket{\lambda_{J_z}}.
    \end{aligned}
\end{equation}
This directly implies
\begin{equation} \label{eq:ladder_propto}
    \begin{aligned}
        J_+\ket{\lambda_{J_z}} &\propto \ket{\lambda_{J_z}+1}, \\
        J_-\ket{\lambda_{J_z}} &\propto \ket{\lambda_{J_z}-1}.
    \end{aligned}
\end{equation}
Before proceeding, we need to clarify an assumption we made to get (\ref{eq:ladder_propto}) that there is no degeneracy
of $\lambda_{J_z}$ in the basis. This result is directly entailed from the assumption that the representation is irreducuble.
To prove this, first assume that there exist $\ket{\lambda_{J_z}}'$ that has the same eigenvalue as $\ket{\lambda_{J_z}}$.
From the properties of $J_i$'s, we can choose the representation such that (\ref{eq:ladder_propto}) is true.
This means that the proper subspace that exclude $\ket{\lambda_{J_z}}'$ is invariant for this choice of representation.
Hence, the representation is reducible which contradict the assumption.

To get the proportionality, consider another operators
\begin{equation}
    J^2 = J_x^2 + J_y^2+J_z^2.
\end{equation}
One can directly chack that the new operator commutes with all $J_i$'s mentioned so far. Furthermore, it is straight
forward that
\begin{equation}
    \begin{aligned}
        J^2 &= J_+J_- + J_z^2 - J_z \\
        &= J_-J_+ + J_z^2 + J_z
    \end{aligned}
\end{equation}
Since $J^2$ commutes with $J_z$, the basis used are also eigenbasis of $J^2$ with eigenvalue $\lambda_{J^2}$. Hence, we
can directly relabel the eigenbasis from $\ket{\lambda_{J_z}}$ to $\ket{\lambda_{J^2}, \lambda_{J_z}}$, and get
\begin{equation}
    \begin{aligned}
        \mel{\lambda_{J^2}, \lambda_{J_z}}{J_+J_-}{\lambda_{J^2}, \lambda_{J_z}} &= \lambda_{J^2} - \lambda_{J_z}(\lambda_{J_z} - 1), \\
        \mel{\lambda_{J^2}, \lambda_{J_z}}{J_-J_+}{\lambda_{J^2}, \lambda_{J_z}} &= \lambda_{J^2} - \lambda_{J_z}(\lambda_{J_z} + 1).
    \end{aligned}
\end{equation}
Since the norm square of any vector in vector space must be non-negative, this requires
\begin{equation}
    \lambda_{J^2} \geq \abs{\lambda_{J_z}}(\abs{\lambda_{J_z}}+1) \geq 0.
\end{equation}
However, the $J_\pm$ makes $\lambda_{J_z}$ unbounded unless the vector vanished at some points, i.e. the equality must 
hold for the upper, and lower bounds of the possible $\lambda_z$'s. If we replace $\lambda_{J^2}$ with $j(j+1)$ where $j$
non-negative. The requirement makes $\lambda_{J_z, \max} = j$, and $\lambda_{J_z, \min} = -j$. Since any pair of $\lambda_{J_z}$'s
are different by an integer, 
\begin{equation}
    \lambda_{J_z, \max} - \lambda_{J_z, \min} = 2j \in \mathbb{Z}_0^+.
\end{equation}
Hence, we can relabel the eigenbasis from $\ket{\lambda_{J^2}, \lambda_{J_z}}$ to $\ket{j, m}$ and choose the proportionality
such that
\begin{equation} \label{eq:ladder}
    \begin{aligned}
        J_+\ket{j, m} &= \sqrt{j(j+1) - m(m + 1)}\ket{j, m+1}, \\
        J_-\ket{j, m} &= \sqrt{j(j+1) - m(m - 1)}\ket{j, m-1},
    \end{aligned}
\end{equation}
where $j \in \{0, 1/2, 1, 3/2, ...\}$, and $m \in \{-j, -j+1, ..., j-1, j\}$. With similar method to the proof for $J_z$,
one can directly prove by contradiction that each irreducible representation must have its unique $j$, i.e., the irreducible
representation can be indexed by $j$. Furthermore, since we have not make any other assumption, $j$ completely identify
every irreducible representation of $\mathfrak{su(2)}$, and all irreducible representation with the same $j$ are isomorphic.

\section{$SU(2)$'s Irreducible Representation}
Consider a representation of $SU(2)$ that is an exponential of $\mathfrak{su(2)}$'s irreducible representations. Let assume
that such representation is reducible. By definition, this means that there is a subspace invariant by any $SU(2)$'s elements.
This further implies that it is also invariant by any $\mathfrak{su(2)}$'s. Since corresponding representation of $SU(2)$,
and $\mathfrak{su(2)}$ are homomorphic, it entails that there is a subspace of $\mathfrak{su(2)}$'s representation that is
invariant to $\mathfrak{su(2)}$, i.e. this specific representation of $\mathfrak{su(2)}$ is reducible, which contradicts
the assumption. Therefore, we can directly promote the irreducible representations of $\mathfrak{su(2)}$ to the one for 
$SU(2)$ by means of matrix exponential along with the same vector space.

\section{$SU(2)$'s Implementation}
In order to implement the representation the basis of the space in which all group operations would act on need to be carefully
selected mainly for the purpose of reducing computational costs. However, it is require the consideration on every symmetry
group to decide. Hence, for now, we are assuming the standard $\{\ket{j, m}\}$ basis. In this basis, the irreducible
representation with index $j$ is the $2j+1$ dimensional space representation. Let the representation of a vector is in 
the ascending order of the basis from $m = -j$ to $m = j$. Hence,
\begin{equation}
    \begin{aligned}
        \left[J_z^{(j)}\right]_{pq} &= (p-j)\delta_{p,q}, \\
        \left[J_+^{(j)}\right]_{pq} &= \sqrt{j(j+1)-(p-j)(q-j)}\delta_{p,q+1}, \\
        J_-^{(j)} &= J_+^{(j)\top}, \\
        J_x^{(j)} &= \left(J_+^{(j)}+J_-^{(j)}\right)/2, \\
        J_y^{(j)} &= -i\left(J_+^{(j)}-J_-^{(j)}\right)/2.
    \end{aligned}
\end{equation}
\end{document}